% OS Software Engineer Intern - 631420
%
% Description
%
% In this position, you will be responsible for researching and analyzing operating system software enhancements for operating systems and virtualization software. You will be working closely with the desktop, server, mobile and phone business units at Intel. Your responsibilities will include but not be limited to: 
% - Analyzing, debugging and evaluating system software and processor errata for current systems and future Intel processors. 
%
% In this role the ideal candidate will have: 
% -Good verbal and written communication skills 
% -Demonstrated skills to analyzing complex problems quickly leading to the development of creative solutions and adapt to a fast paced environment
% 
% Qualifications
% 
% Requirements: 
% -MS in computer science or a BS degree with relevant work experience. 
% -One or more years of work experience or coursework to demonstrate knowledge regarding processor architecture, operating systems internals, kernel debug skills, device driver development, performance analysis tools and techniques. 
% -One or more years work experience or coursework to demonstrate expertise on one or more of the following areas: power management, virtualization, RAS, graphics, sensor and security 
% -One or more years of work experience or coursework on low-level hardware programming and debugging expertise.
% 
% Job Category: Software Engineering
% 
% Primary Location: USA-Oregon, Hillsboro
% 
% Full/Part Time: Full Time
% 
% Job Type: Student/Intern
% 
% Regular/Temporary: Regular
% 
% Posting Date: Oct 21, 2012
% 
% Apply Before: Oct 22, 2013
%  
% Business Group
% 
% The Intel Software and Services Group (SSG) connects Intel to the worldwide software community. SSG strives to bring competitive advantage to Intel platforms by helping independent software vendors, operating system developers, OEMs, channel members and systems integrators deliver exceptional customer value and achieve differentiation on Intel processor technologies. SSG provides global leadership to the software community through its technical expertise, industry enabling activities, and developer products and programs.
% 
% Posting Statement: We will accept applications/resumes until 60 days after posting date or earlier at Intel’s discretion. Intel invites people of all ages currently enrolled in an academic institution (or graduated within the last 18 months) to apply.
 

\documentclass[10pt,letterpaper]{article}
\usepackage[letterpaper,margin=0.75in]{geometry}
\usepackage[utf8]{inputenc}
\usepackage{mdwlist}
\usepackage[T1]{fontenc}
\usepackage{textcomp}
% \usepackage{tgpagella}
\pagestyle{empty}
\setlength{\tabcolsep}{0em}

% indentsection style, used for sections that aren't already in lists
% that need indentation to the level of all text in the document
\newenvironment{indentsection}[1]%
{\begin{list}{}%
	{\setlength{\leftmargin}{#1}}%
	\item[]%
}
{\end{list}}

% opposite of above; bump a section back toward the left margin
\newenvironment{unindentsection}[1]%
{\begin{list}{}%
	{\setlength{\leftmargin}{-0.5#1}}%
	\item[]%
}
{\end{list}}

% format two pieces of text, one left aligned and one right aligned
\newcommand{\headerrow}[2]
{\begin{tabular*}{\linewidth}{l@{\extracolsep{\fill}}r}
	#1 &
	#2 \\
\end{tabular*}}

% make "C++" look pretty when used in text by touching up the plus signs
\newcommand{\CPP}
{C\nolinebreak[4]\hspace{-.05em}\raisebox{.22ex}{\footnotesize\bf ++}}

% and the actual content starts here
\begin{document}

\begin{center}
{\LARGE \textbf{Derek Qian}}

1245 SW Grover St\ \ \textbullet
\ \ Apt\ 303\ \ \textbullet
\ \ Portland, OR 97239
\\
(503) 267-5036\ \ \textbullet
\ \ derek.dejun@gmail.com
\end{center}

% summary
\hrule
\vspace{-0.4em}
\subsection*{Objective}

\begin{indentsection}{\parindent}
\hyphenpenalty=1000
To obtain a system software engineer internship at Intel. With the knowlege of computer architecture and hardware devices, and the work experience on low-level hardware programming, bootloader porting, kernel debugging and driver development, I want to get more trainning and gain experince on OS software develpment.
\end{indentsection}

% summary
\hrule
\vspace{-0.4em}
\subsection*{Summary}

\begin{indentsection}{\parindent}
\hyphenpenalty=1000
Derek is seeking a system software internship beginning after March 20, 2013. With the knowlege of computer architecture and operating system internal and the work experience on the porting of Embedded Operating System, including bootloader, kernel and device drivers, he wants to gain more experience on OS software enhancement technique. He is interested in operating systems and virtualization software. After getting his Master's Degree in Electrical and electronics Engineering and over 5 years' hand-on experience on computer hardware and software, Derek felt it necessary to solidate his software skill, this is why he is pursuing his Master's Degree in Computer Science at Portland State University now. This intern job is the one he is preparing for and hunting for.

He is a perfect candidate on this job. He has a solid knowledge. His interest focus on system software development. He has worked in mobile computation area for over 5 years, developing PDA (Portable Digital Assistant), PND (Portable Navigation Device), PMC (Portable Media Center) and Smartphone. He has a good understanding of general PC architecture and master the knowledge of embedded development, hardware interaction (register and MMIO), and prototype platform development environments. He is proficient in C/C++ and object oriented programming and knows system-level debugging very well.

During his past work, he experienced software development process and agile SW product development methodology. He knows protocols like USB, TCP/IP, I2C, SPI, RS232, and experienced driver port and development for devices like NOR/NAND FLASH, LCD, TOUCH SCREEN, USB, UART, AUDIO, KEYPAD, SD/MMC/SDIO for Windows/Linux system as well as bootloader and kernel development. He can read schematics and PCB diagrams. He is a good team player and can work in a small team, however, he also has ability to operate independently.
\end{indentsection}

% skill
\hrule
\vspace{-0.4em}
\subsection*{Skills}

\begin{indentsection}{\parindent}
\hyphenpenalty=1000
\begin{description*}
	\item[Languages:]
	C, \CPP, Java, x86/ARM assembly, makefile, shell script, Python, SQL, \LaTeX
	\item[Tools:]
	gcc, arm-gcc, emacs/vim, git/svn, Visual Studio, Platform Builder, Eclipse, PADS
	\item[Open source projects:]
	Android, linux kernel, u-boot, busybox, buildroot, poppler
\end{description*}
\end{indentsection}

% education
\hrule
\vspace{-0.4em}
\subsection*{Education}

\begin{itemize}
	\parskip=0.1em

	\item 
	\headerrow
		{\textbf{Portland State University}}
		{\textbf{Portland, OR}}
	\\
	\headerrow
		{\emph{College of Engineering \& Computer Science, M.S. Computer Science}}
		{\emph{2011 -- present}}
	\begin{itemize*}
		\item \textbf{GPA:} 3.91/4.0.
		\item \textbf{Main Courses:} 
                                Software Engineering, 
                                Building Software Systems with Components,
                                Programming Languages, 
                                Computer Architecture,
                                Parallel Progamming, 
                                Concepts of Operating Systems, 
                                Internetworking Protocols, 
                                Artifical Intelligence,
                                Theory of Computation.
	\end{itemize*}

	\item 
	\headerrow
		{\textbf{Southeast University}}
		{\textbf{Nanjing, Jiangsu, China}}
	\\
	\headerrow
		{\emph{PhD candidate, Electrical and Electronics Engineering}}
		{\emph{2008 -- 2011}}
	\begin{itemize*}
		\item \textbf{GPA:} 3.67/4.0.
		\item \textbf{Main Courses:} 
                                System on Chip (SoC) Design, 
                                RF Microelectronics, 
                                ULSI Devices Circuit \& System, 
                                Introduction to Microactuators, 
                                Mathematical Model.
	\end{itemize*}

	\item 
	\headerrow
		{\textbf{Southeast University}}
		{\textbf{Nanjing, Jiangsu, China}}
	\\
	\headerrow
		{\emph{M.S. Electrical and Electronics Engineering}}
		{\emph{2004 -- 2007}}
	\begin{itemize*}
		\item Top 10\%.
                \item \textbf{Main Courses:} 
                                Hardware \& Software of Microcomuter System, 
                                Advanced Digital Signal Processing, 
                                Numerical Analysis, 
                                Elements of Information Technology,
                                VLSI System Design,
                                Application Specific Intergrated Circuit Design, 
                                Design of Anolog CMOS Integrated Circuits.
                                % Physics of VLSI,
                                % Physics of Semiconductor Devices.
	\end{itemize*}

	\item 
	\headerrow
		{\textbf{Southeast University}}
		{\textbf{Nanjing, Jiangsu, China}}
	\\
	\headerrow
		{\emph{Bachelor's degree, Electrical and Electronics Engineering}}
		{\emph{2000 -- 2004}}
	\begin{itemize*}
		\item Top 10\%.
                \item \textbf{Main Courses:}
                                Programming \& C Language,
                                Data Structures \& Algorithm,
                                C++ Software Engineering,
                                Windows Programming,
                                Operating System,
                                Fundamentals of Computer Network,
                                Microcomputer System \& Interface,
                                % Microcomputer Experiment,
                                % Graph Theory \& Its Application,
                                Fundamentals of Circuit,
                                Signals \& Linear Systems,
                                Fundamentals of Electronics Circuits,
                                Computer Architecture \& Logic Design,
                                % Laboratory of Circuit \& Logic Design,
                                Principle \& Application of MCU,
                                Principle \& Application of Programmable Devices,
                                Digital Signal Processing,
                                Principle of Communication,
                                Principles of Automatic Control,
                                MATLAB Language,
                                CAD for Printed Circuite Board,
                                Electromagnetic Theory,
                                Fields \& Waves in Information Electric Technique,
                                % Conspectus of Micro-electronic Devices,
                                Analysis \& Design of Digital Integrated Circuits,
                                Analysis \& Design of Analog Integrated Circuits,
                                Integrated Circuits CAD.
                                % Integrated Circuits Manufacturing Techniques,
                                % Reliability of Microelectronic Devices \& Integrated Circuits,
                                % Fundamentals of Optoelectronic Physics,
                                % Numerical Method of Partial Differential Equations,
                                % Probability \& Statistics,
                                % Practical Statistical Analysis,
                                % Numerical Computing \& Modeling,
                                % Mathematical Model,
                                % Mathematical Programming,
                                % Equations of Mathematical Physics,
                                % Fuzzy Mathematics,
                                % Digital Systems Design \& Electronic Technique Experimentation,
                                % Comprehensive Electronic Design, % experiment
                                % Course Design for Integrated Circuits,
                                % Metal Working Practice,
                                % Engineering Chemistry,
                                % Graduation Project,
                                % The principle of Color TV,
                                % Computer Graphics, % AutoCAD
                                % Advanced Mathematics,
                                % Geometry \& Algebra.
	\end{itemize*}

\end{itemize}


\hrule
\vspace{-0.4em}
\subsection*{Experience}

\begin{itemize}
	\parskip=0.1em

	\item
	\headerrow
		{\textbf{Portland State University}}
		{\textbf{Portland, OR}}
	\\
	\headerrow
		{\emph{Student \& Teaching Assistant}}
		{\emph{2011 -- present}}
	\begin{itemize*}
		\item Course Project: CrazyPuzzle, a math puzzle game for Android platform. (Eclipse, Java, Android)
		\item Course Project: ftpread, a simple ftp client working in both passive
                and active mode. (FTP, TCP/IP, C, socket)
		\item Course Project: projects from parallel programing course. (Linux, pthread, OpenMP, MPI, python)
		\item Course Project: Arum, a lightweight, easy-to-use, all-in-one solution
                for colleting a variety of performance data of applications. (Dyninst, C/C++, Linux, GNU Make, \LaTeX)
		% \item Research Project: coDoc, a tool managing code and its spec. (Eclipse, Java,
                % C++, JNI, SWT, RCP, CDT, poppler, Cario)
		\item Teach Assistant: Data Structures, Computer Systems Programming,
		Discrete Structures, Elements Of Software Engineering.
	\end{itemize*}

	\item
	\headerrow
		{\textbf{Skypine Co., Ltd.}}
		{\textbf{www.skypine.cn}}
	\\
	\headerrow
		{\emph{Senior Software Engineer}}
		{\emph{2011 -- 2011}}
	\begin{itemize*}
		\item The core board for Navigation Unit using i.MX51 processor and Android 2.3. (u-boot, Embedded Linux, 
                Android Framework, Driver)
		\item The core board for Navigation Unit using i.MX51 processor and Windows CE 6.0. (Visual Studio, BSP, 
                Bootloader, Kernel, Driver)
		\item Optimized the media player. (Direct Draw, Direct Show, Overlay, YUV, RGB)
		\item Customized the touchscreen calibrator.
	\end{itemize*}

	\item
	\headerrow
		{\textbf{Seuic, LLC.}}
		{\textbf{www.seuic.com}}
	\\
	\headerrow
		{\emph{Senior Software Engineer \& Project Manager}}
		{\emph{2004 -- 2010}}
	\begin{itemize*}
		\item PMC (Personal Media Center), using CS8900 and Android 2.3. (Android, hal, u-boot, linux kernel, device driver)
		\item PDA (Personal Digital Assistant), using PXA 255 and Embedded Linux. (u-boot, linux kernel, device driver)
		\item Managed the project developing a smartphone using PXA310 processor
                and Windows CE6.0. (XSCALE, OAL, MS Project, WTL, MFC)
		% \item Smartphone using PXA270 and Windows Mobile 6.04.
		\item PND (Portable Navigation Device), using PXA 270 and Windows CE 4.2. (NMEA, MLC NAND flash, FMD, FAL, ECC)
	\end{itemize*}

\end{itemize}


\hrule
\vspace{-0.4em}
\subsection*{Publications}

\begin{itemize*}
	\item
        \textbf{Qian, D.}, Zhang, Z. The Parsing of NMEA0183 Protocol. \emph{Chinese Journal of ELECTRON DEVICES, v 30, n 2, January, 2007, p 698-701}
        \item
        Zhang, Z., \textbf{Qian, D.}, Zhou, Q., Lu, H., Hu, C. A real-time low-power strategy based on extended slack time reclaiming algorithm. \emph{Chinese Journal of Circuits and Systems, v 14, n 1, January, 2009, p 18-22}
        \item
        Zhang, Z., \textbf{Qian, D.}, Hu, C. An efficient low-power scheme based on probability distribution of system workloads. \emph{Chinese Journal of Circuits and Systems, v 15, n 6, June, 2010, p 18-22}
        \item
        Zhang, Z., Chen, X., \textbf{Qian, D.}, Hu, C. Dynamic Voltage Scaling for Real-Time Systems with System Workload Analysis. \emph{IEICE Transactions 93-C(3): 399-406 (2010)}
	\item
        \textbf{Qian, D.}, Zhang, Z., Tian, X., Hu, C. Low power scheduling for periodic real-time systems with Dynamic Voltage Scaling processor. \emph{Computer Application and System Modeling (ICCASM), 2010 International Conference on, vol.11, pp.V11-244-V11-248, 22-24 Oct. 2010}
	\item
	\textbf{Qian, D.}, Zhang, Z., Hu, C., Ji, X. Energy-Aware Task Scheduling for Real-Time Systems with Discrete Frequencies. \emph{IEICE Transactions 94-D(4): 822-832 (2011)}
\end{itemize*}


\end{document}
