%________________________________________________________________________________________
% @brief    LaTeX2e Resume for Kamil K Wojcicki
% @author   Kamil K Wojcicki
% @url   http://linux.dsplabs.com.au/?p=54
% @date     Decemebr 2007
% @info     Based on Latex Resume Template by Chris Paciorek 
%        http://www.biostat.harvard.edu/~paciorek/
%________________________________________________________________________________________
\documentclass[margin,line,a4paper]{resume05}

\usepackage[latin1]{inputenc} %utf8
\usepackage[english,danish]{babel}
\usepackage[T1]{fontenc}
\usepackage{graphicx,wrapfig}
\usepackage{url}
\usepackage[colorlinks=true, a4paper=true, pdfstartview=FitV,
linkcolor=blue, citecolor=blue, urlcolor=blue]{hyperref}
\pdfcompresslevel=9


\begin{document}
{\sc \Large Curriculum Vitae -- Thomas R. N. Jansson}
\begin{resume}
    \vspace{0.5cm}
    \begin{wrapfigure}{R}{0.6\textwidth}
        \vspace{-1cm}
       \begin{center}
       \includegraphics[width=0.6\textwidth]{me}
       \end{center}
        \vspace{-1cm}
    \end{wrapfigure}

    \section{\mysidestyle Personal\\Information}%\vspace{2mm}
    Thomas R. N. Jansson \\
    Allersgade 18, 2 tv.  \\ 
    2200 Copenhagen N \\ 
    Denmark \\ 
    tel: +45 29722392 \\
    \href{mailto:tjansson@tjansson.dk}{tjansson@tjansson.dk} \\
    \href{http://www.tjansson.dk}{www.tjansson.dk}\\    

    I was born and raised in Copenhagen where I have liv\-ed all my life except
    6 months in which I studied on the Arctic island Svalbard and currently I
    am living together with my girlfriend Pernille Petersen (claissical flute
    player).  I have always been very interested in computers and science and
    by studying physics I had the possibility to combine these two subjects. 

    Currently I work at Schlumberger as a geophysicist and Linux system
    administator, which is a quite ideal combination in my opinion. A typical
    day of work could either be spent doing wavelet extraction on seismic data
    or writing a munin plugin to various services on the servers. 
    
    When I am not doing physics or computers I  enjoy sports. I play badminton
    once a week and fitness twice a week. Beside sports my hobbies are
    photography and traveling.

    \section{\mysidestyle Education} \textbf{Masters degree in geophysics from
    the University of Copenhagen} (2006-2008). Thesis advisors: Klaus Mosegaard
    (KU) and Trine Dahl Jensen (GEUS).  Thesis title: \textit{Receiver function
    modeling}. Modeling local subsurface velocity structures using multiple
    diverse algorithms.

    \textbf{Bachelor degree in physics from the University of Copenhagen}
    (2001-2006).  Thesis advisor: Tomas Bohr (DTU Physics).
    Thesis title: Symmetry breaking in the free surface of rotating fluids
    with high Reynolds numbers.  Enrolled: September 2001

    \textbf{Rysensteens Gymnasium} (1998-2001) High school.  I attended a
    special mathematical/physical line. Located in Copenhagen.

    \textbf{Den Classenske Legatskole} (1989-1998) Municipal School.
    

\section{\mysidestyle Job experiences}\vspace{1mm}
\begin{description}
    
    \item[2009 April$\rightarrow$ ] Employed as Inversion Geophysicist at
    Schlumberger in Copenhagen. I am in the NSG (North Sea Geomarket), DCS
    (Data and Consulting Services) under RSS (Reservoir Seismic Services). Work
    tasks include inversion of seismic and well data for clients. Beside the
    geophysical work I also fill the role as Linux systems administration for
    selected Linux servers. I have been maintaining both roles 50/50 since I
    started. IT wise I am looking after 40 servers (and 3 NetApp's) in
    Scotland, Norway and Denmark controlled with NIS and LDAP authentication.

    \item[2009 January (Thomas Jansson IT)] Constructed web frontend for 
    the ``Shallow Water Model'' for use in teaching at the geophysical
    department of University of Copenhagen. Referee: Eigil Kaas
    (\href{mailto:kaas@gfy.ku.dk}{kaas@gfy.ku.dk}) and Aksel Walløe Hansen
    (\href{mailto:awh@gfy.ku.dk}{awh@gfy.ku.dk}).
    \item[2008 August $\rightarrow$ October (Thomas Jansson IT)] Gave a one-day course 
    in the use of the content management system Drupal for DTM International
    A/S. Subsequently employed as a consultant. 
    \item[2008 July (Thomas Jansson IT)] Building website for "First Workshop on Satellite Imaging
    of the Arctic", see \url{www.gfy.ku.dk/~awh/satellite-imaging/}. 
    \item[2008 June (Thomas Jansson IT)] Constructed web frontend for 
     the ``Simple Meridional Energy Balance Model'' for use in teaching at the geophysical
    department of University of Copenhagen. Referee: Eigil Kaas
    (\href{mailto:kaas@gfy.ku.dk}{kaas@gfy.ku.dk}). see
    \url{http://gfy.ku.dk/~kaas/onedmodel/run.php}. 
    \item[2008 January $\rightarrow$ October (Thomas Jansson IT) ] Further development of the python-
    based graphical user interface, pyGravsoft. This time the program was
    tested in Malaysia and included user surveys. 
    \item[2007 June (Thomas Jansson IT) ] A PHP/MySQL based web page for the magazine Kvant, see
    \url{www.kvant.dk}. The new site has a searchable index of every
    article in Kvant. 
    \item[2007 May (Thomas Jansson IT) ] Building a python based graphical user interface to a text
    based gravimetric program called Gravsoft, see
    \url{www.gfy.ku.dk/~cct/}. Referee: Professor Carl Christian Tscherning
    (\href{mailto:cct@gfy.ku.dk}{cct@gfy.ku.dk}).
    \item[2007 May (Thomas Jansson IT)] Building a Xoops (CMS) based web page for LJ Ejendomme, see
    \url{www.lj-ejendomme.dk}.
    \item[2006 August $\rightarrow$ 2009 Feburary (Thomas Jansson IT)] Unix system administrator at the Geological
    Institute, University of Copenhagen. Work tasks: Normal system
    administration of SUN and Linux servers as well as setting up a small Linux
    cluster for sea modeling. Referee: Professor Hans Thybo,
    \href{mailto:thybo@geo.ku.dk}{thybo@geo.ku.dk}. 
    \item[2006 July (Thomas Jansson IT)] Building and setup of a Linux based file and applications
    server for Utopiarejser.  
    \item[2006 March (Thomas Jansson IT)] Building a web page for ``Copenhagen Global Change
    Initiative'', see \url{www.klima.nbi.dk}. 
    \item[2006 January $\rightarrow$ ] Started a company:\textit{Thomas Jansson IT}.
    I am doing consulting jobs as a programmer, system administrator, server
    building and web design. In this connection I write IT articles on my blog 
    \url{www.tjansson.dk}.
    \item[2005 January $\rightarrow$ 2005 July ] Substitute math and physics teacher at Bjørns
    international school (elementary school). 
\end{description}


\section{\mysidestyle Courses taken}\vspace{1mm}
\begin{description}
    \item[ 6-jun-2010 $\rightarrow$ 25-jun-2010] Schlumberger DCS DeepBlue 2 -
    Inversion Geophysics and Advanced Sesimic Inversion Techniques \\
    This course will introduce the student to the concepts of seismic inversion and
    the various methods and algorithms available. Second half covers Advanced
    Sesimic Inversion Techniques. 

    \item[ 23-nov-2009 $\rightarrow$ 27-nov-2009] RH300 RHCE Rapid Track Course and RHCE Exam.\\
    The Red Hat Certified Engineer course is designed for UNIX- and
    Linux-experienced users, networking specialists, and system administrators.
    This 5-day course provides intensive hands-on training on Red Hat
    Enterprise Linux 5, and includes the RHCE Certification Lab Exam on day 5.

    \item[7-aug-2009 $\rightarrow$ 27-sep-2009 ] Schlumberger DCS DeepBlue 1 Reservoir
    Characterization course. \\
    The course is designed as a basic introduction to integrated
    reservoir characterization. The course content covers the fundamentals of
    Petrophysics, Sonic waveforms, Borehole geology, Reservoir geology and
    Seismic. Included is the practical aspects of analysis of data from
    distinct sources and disciplines for characterization. Ultimately the
    integration of this data and mapping for reservoir visualization. Software
    used Geoframe and Petrel. 
    
    \item[ 29-jun-2009 $\rightarrow$ 03-jul-2009] Data ONTAP Fundamentals.\\
    This 5-day instructor-led course introduces basic support and
    administrative functions of the Data ONTAP operating system. The course
    emphasizes core facts and concepts of a NetApp Storage System. The elements
    of the Data ONTAP operating system covered in this course are Write
    Anywhere File Layout (WAFL) file system, volumes, aggregates, qtress, and
    quotas. Hands-on labs for the course focus on the basic administrative use
    of Data ONTAP in NAS and IP-SAN environments.
\end{description}


\section{\mysidestyle Extracurricular activities}\vspace{1mm}
    \begin{description}
    \item[2008 July 21 $\rightarrow$ 1 August ] Attended the ESA sponsored 
    summer school in Alpbach, Austria. The subject was "Sample Return from
    Moon, Asteroids and Comets".
    \item[2007 December $\rightarrow$ ] Editor at Kvant.  Kvant is the members
    magazine for ``Dansk Fysisk Selskab'', ``Astronomisk Selskab'', ``Selskabet
    for Naturlærens Udbredelse'' and ``Dansk Geofysisk Forening''. 3000 copies
    four times a year. 
    \item[2006 August $\rightarrow$ December ] Studies at UNIS, Svalbard. I
    studied oceanography and remote sensing for one semester at Svalbard,
    78$^\circ$ N.
    \item[2001 December $\rightarrow$ 2008 June ] Editor at \emph{Gamma}.
    \emph{Gamma} is student-operated magazine sponsored by the Niels Bohr
    Institute. 3000 copies four times a year. Functioned as editor in chief
    in several periods. 
    \item[2007 February $\rightarrow$ 2008 September ] Board member in the
    geophysical student union.
    \end{description}






\section{\mysidestyle Publications}
    Thomas R. N. Jansson, Martin P. Haspang, Kåre H. Jensen, Pascal
    Hersen, and Tomas Bohr, \textit{Polygons on a Rotating Fluid
    Surface}, Physical Review Letters \textbf{96} 174502 (2006).
    \url{doi:10.1103/PhysRevLett.96.174502} 
    
    The article was the continued work of my bachelors project. The
    article made quite a buzz and was cited in news medias such as Nature
    and the New York Times. See\\
    \href{http://www.nature.com/news/2006/060515/full/news060515-17.html}{www.nature.com/news/2006/060515/full/news060515-17.html}\\
    \href{http://tierneylab.blogs.nytimes.com/2007/04/05/and-saturns-hexagon-shall-be-called/}{tierneylab.blogs.nytimes.com/2007/04/05/and-saturns-hexagon-shall-be-called/}
    
    Janni Nielsen, Thomas R. N. Jansson, Carl C. Tscherning, \emph{Creating a user
    interface to GRAVSOFT}, Report for Klang Valley Height Modernisation
    Project, 2008. 
    
    J.Nielsen, C.C.Tscherning, T.R.N.Jansson, R.Forsberg,
    \emph{Development of a Python interface to the GRAVSOFT gravity
    fiel programs}, proceedings of IAG 2009 Scientific Assembly
    "Geodesy for Planet Earth", Springer. 
    
    Thomas R. N. Jansson, \emph{En problemorienteret introduktion til Linux}. A
    130-page guide to problem solving in Linux and open source programs. Licensed
    under \textit{Open document license} can be found at \url{www.tjansson.dk/?page_id=4}    


\section{ \mysidestyle Conference contributions}
    R. Forsberg. T. R. N. Jansson, J. E. Nielsen, C. C. Tscherning,
    \emph{Development of a Python interface to the GRAVSOFT gravity field
    programs.}, IAG 2009 - Geodesy for Planet Earth, Buenos Aires
    (September 2009).  

    D. G. Bennett, H. Changela, N. Dalcher, C. L. Goldmann, M. Heger, T.
    Hiriart, T. R. N. Jansson, S. Kern, K. Motamedi, M. Petitat, G.
    Sangiovanni, J. Spurmann, A. Stiegler, M. Unterberger, E. Vigren.
    \emph{I. T. -- R. O. C. K. S. Comet Nuclei Sample Return Mission}, International
    Astronautical Congress September 2008, Glassgow. 

    Tomas Bohr, Pascal Hersen, Thomas R. N. Jansson, Martin P. Haspang and K. H.
    Jensen, \emph{Polygons on a Rotating Fluid Surface}, The 6th Euromech Fluid Mechanics
    Conference, Stockholm (June 2006)

    T.R.N. Jansson, M.P. Haspang, K.H. Jensen, P. Hersen \& T. Bohr,
    \emph{Polygons on a
    Rotating Fluid Surface}, Second International Symposium on Instability and
    Bifurcations in Fluid Dynamics, Technical University of Denmark (August 2006)



\section{ \mysidestyle Presentations}
The following presentations was can be found on \url{www.tjansson.dk}. All the
presentation was held at the University of Copenhagen. 
\begin{itemize}
    \item \emph{Receiver function modellering}, Geofysikdag held by ``Dansk
    Geofysisk Forening'',  11 April 2008. 
    \item \emph{Dæmpning af seismiske bølger}, Hovedfagskollokvium, 21 December
    2007. 
    \item \emph{Et bachelorprojekt om roterende vand}, inspirational talk for
    new bachelor students, 9 November 2007.
\end{itemize}

\section{\mysidestyle Selected popular science articles}
As mentioned earlier I was the editor at Gamma for 7 years and later started at
Kvant. During this period I have written 24 small articles and news stories on
physics and technology. The articles can be found on the pages:
\url{www.kvant.dk} and \url{www.gamma.nbi.dk}. Where no other authors are
stated I am the author.

\begin{itemize}
    \item \emph{Review: ``Kvantespring i det 20. århundrede''}, Gamma,
    fall, 2008.
    \item \emph{Review: ``Insultingly stupid movie physics''}, Kvant 3,
    2008.
    \item \emph{Eksperiment med flydende metaller relateret til jordens
    magnetfelt}, Gamma 145, 2007.
    \item \emph{Ru vingeoverflade kan spare brændsel}, Gamma 141, 2006.
    \item Kåre H. Jensen og Thomas R. N. Jansson, \emph{Open source programmer til
    videnskabelig brug}, Gamma 140, 2005.
    \item Alexandru Nicolin, Thomas R. N. Jansson og Andreas Lemark,
    \emph{Interview med Nobelpristager David J. Gross}, Gamma 138, Maj 2005.
    \item \emph{Review: ``Fra superstrenge til stjerner''},
    Gamma 132, 2003.
\end{itemize}

    
    

\section{\mysidestyle Computer skills}\vspace{1mm}
\begin{description}
\item[Operating systems] Advanced experience with the most flavors of Linux, Ubuntu,
    Debian, CentOS, Mandriva and Rocks Cluster Linux. Experienced with Sun
    Solaris 5.7 $\rightarrow$ 5.9, Microsoft Windows and to some extent Mac OS
    X which is very *nix like.
    \item[Servers and databases] Apache2, munin, openssh, subversion, NFS, CUPS, MySQL.
    \item[CMF, CMS and CMS-like systems] Xoops, Wordpress, Drupal, Limesurvey.
    \item[Programming, scripting and markup languages] Python, Bash and
    tcsh (daily). PHP, \LaTeXe, HTML, CSS, matlab (Often). C++ and Fortran (seldomly).
    \item[Courses] Attended 5 days NetApp course, 5 days RHCE Rapid Track Course.  
    \item[Certifications] Red Hat Certified Technician.  
    \item[Open source projetcs] Co-author and owner of the python based open
    source project Sinthgunt.  An easy python/GTK frontend to ffmpeg using more
    than 100 pre-configured conversion settings. Included in the repositories
    of various Linux distributions.\\ \url{http://code.google.com/p/sinthgunt/}
\end{description}
    

\section{\mysidestyle Language skills}
    My mother tongue is Danish, but almost everything I write is in English both in
    connection to computers in general on my blog \url{www.tjansson.dk} and in
    scientific work. \textbf{Danish}: Native tongue. \textbf{English}: Fluent.
    \textbf{German}: High-school level.

\end{resume}
\end{document}
